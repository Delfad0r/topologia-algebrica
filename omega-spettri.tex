\section*{\texorpdfstring{$\Omega$}{Omega}-spettri e coomologia}

\subsection*{Definizione}
\begin{frame*}
\begin{definition}
Un \emph{$\Omega$-spettro} è una famiglia di CW-complessi $\{K_n\}_{n\ge 1}$ dotata di equivalenze omotopiche deboli
\[
\map{\theta_n}{K_n}{\Omega K_{n+1}}.
\]
\end{definition}
\begin{itemize}
\item È possibile estendere la famiglia anche a indici $n\le 0$: è sufficiente considerare come $K_{n-1}$ un'approssimazione CW di $\Omega K_n$.
\end{itemize}
\begin{proposition}
Siano $\map{f}{Y}{Z}$ un'equivalenza omotopica debole, $X$ un CW-complesso. Allora la composizione
\[
\map[\iso]{f\circ-{}}{\hmaps{X,Y}}{\hmaps{X,Z}}
\]
è una biiezione.
\end{proposition}
\end{frame*}

\subsection*{Struttura di gruppo}
\begin{frame*}
D'ora in poi, tutti gli spazi di cui parleremo saranno CW-complessi puntati.
\begin{itemize}
\item Possiamo dare a $\hmaps{X,K_n}$ una struttura di gruppo imponendo che
\[
\map[\iso]{\theta_n\circ-{}}{\hmaps{X,K_n}}{\hmaps{X,\Omega K_{n+1}}}
\]
sia un isomorfismo di gruppi.
\item In questo modo, $\hmaps{X,K_n}$ risulta essere un gruppo abeliano:
\begin{diagram}[column sep=huge]
\&\hmaps{\Sigma X,K_{n+1}}\rar["\iso"']{\theta_{n+1}\circ-}\dar{\iso}\&\hmaps{\Sigma X,\Omega K_{n+2}}\dar{\iso}\\
\hmaps{X,K_n}\rar["\iso"']{\theta_n\circ-}\&\hmaps{X,\Omega K_{n+1}}\rar{\Omega\theta_{n+1}\circ-}\&\hmaps{X,\Omega^2 K_{n+2}}\nospacepunct{.}
\end{diagram}
\item Di conseguenza, per ogni $n\in\ZZ$ otteniamo il funtore
\[
\map{\hmaps{-,K_n}}{\CW0}{\Ab^{\text{op}}}.
\]
\end{itemize}
\end{frame*}

\subsection*{Teorie coomologiche}
\begin{frame*}
\begin{theorem}
Sia $\{K_n\}_{n\in\ZZ}$ un $\Omega$-spettro. Allora i funtori $h^n=\hmaps{-,K_n}$ definiscono una teoria coomologica ridotta sulla categoria \CW0.
\end{theorem}
\begin{reminder}
Una \emph{teoria coomologica ridotta} sulla categoria \CW0 è una famiglia di funtori
\begin{align*}
\map{h^n}{\CW0}{\Ab^{\text{op}}},&&n\in\ZZ
\end{align*}
che soddisfa i seguenti assiomi.
\begin{enumerate}
\item Per ogni coppia $(X,A)$ con $x_0\in A$ esiste una successione esatta lunga
\begin{diagram}[column sep=small]
\ldots\rar{\delta}\&h^n(X/A)\rar{q^*}\&h^n(X)\rar{i^*}\&h^n(A)\rar{\delta}\&h^{n+1}(X/A)\rar{q^*}\&\ldots
\end{diagram}
naturale nella coppia $(X,A)$.
\item\label{cohomology-axiom-wedge} Per ogni famiglia $\{X_\alpha\}_{\alpha}$, le inclusioni inducono un isomorfismo
\[
\umap{h^n\left(\textstyle\bigvee_\alpha X_\alpha\right)}{\textstyle\prod_\alpha h^n(X_\alpha)}.
\]
\end{enumerate}
\end{reminder}
\end{frame*}

\subsection*{Dimostrazione}
\begin{frame*}
\begin{itemize}
\item [superfluo]Per ogni $\map{f}{X}{Y}$,
\[
\map{h^n(f)=(-\circ f)}{\hmaps{Y,K_n}}{\hmaps{X,K_n}}
\]
è un omomorfismo di gruppi:
\begin{diagram}
\hmaps{Y,K_n}\rar{-\circ f}\dar{\iso}\&\hmaps{X,K_n}\dar{\iso}\\
\hmaps{Y,\Omega K_{n+1}}\rar{-\circ f}\&\hmaps{X,\Omega K_{n+1}}\nospacepunct{.}
\end{diagram}
\item L'assioma \ref{cohomology-axiom-wedge}. è soddisfatto: nella categoria \CW0 un morfismo
\[
\umap{{\textstyle\bigvee_\alpha X_\alpha}}{K_n}
\]
è precisamente una collezione di morfismi
\[
\{\umap{X_\alpha}{K_n}\}_\alpha.
\]
\end{itemize}
\end{frame*}

\begin{frame*}
\begin{onlyenv}<1>
Costruiamo la successione esatta lunga associata a una coppia $(X,A)$.
\begin{diagram}[column sep=11pt]
A\rar[hook]\dar[equal]\&X\rar[hook]\dar[equal]\&X\cup\cone{A}\rar[hook]\dar{\iso}\&(X\cup\cone{A})\cup\cone{X}\rar[hook]\dar{\iso}\&\big((X\cup\cone{A})\cup\cone{X}\big)\cup\cone(X\cup\cone{A})\dar{\iso}\\
A\rar[hook]\dar\&X\rar\dar\&X/A\rar\dar\&\Sigma A\rar[hook]\dar\&\Sigma X\dar\\
B\rar[hook]\&Y\rar\&Y/B\rar\&\Sigma B\rar[hook]\&\Sigma Y
\end{diagram}
\end{onlyenv}

Applicando il funtore $h^n$, otteniamo una successione di gruppi abeliani:
\begin{diagram}
\hmaps{A,K_n}\&\hmaps{X,K_n}\lar\&\hmaps{X/A,K_n}\lar\&\hmaps{\Sigma A,K_n}\lar\&\hmaps{\Sigma X,K_n}\lar.
\end{diagram}

\begin{onlyenv}<1-2>
Per verificarne l'esattezza, è sufficiente mostrare che
\begin{diagram}
\hmaps{A,K_n}\&\hmaps{X,K_n}\lar\&\hmaps{X\cup\cone{A},K_n}\lar
\end{diagram}
è esatta.\end{onlyenv} \begin{onlyenv}<2>Sia $\map{f}{X}{K_n}$. Allora
\begingroup
\addtolength\jot{.15em}
\begin{align*}
&\text{$f$ appartiene al nucleo di $\umap{\hmaps{X,K_n}}{\hmaps{A,K_n}}$}\\
\iff&\text{$f|_A$ è omotopicamente banale (fissando il punto base)}\\
\iff&\text{$f$ si estende a una mappa $\umap{X\cup\cone{A}}{K_n}$}\\
\iff&\text{$f$ appartiene all'immagine di $\umap{\hmaps{X\cup\cone{A},K_n}}{\hmaps{X,K_n}}$}.
\end{align*}
\endgroup
\end{onlyenv}

\begin{onlyenv}<3>
Tale successione è esatta e naturale. Posto $K=K_n,K'=K_{n+1}$, le due successioni corrispondenti si possono incollare.
\begin{diagram}[column sep=10pt]
\&\&\hmaps{A,K}\dar{\iso}\&\hmaps{X,K}\lar\dar{\iso}\&\hmaps{X/A,K}\lar\&\hmaps{\Sigma A,K}\lar\\
\&\&\hmaps{A,K'}\dar{\iso}\&\hmaps{X,K'}\lar\dar{\iso}\\
\hmaps{X,K'}\&\hmaps{X/A,K'}\lar\&\hmaps{\Sigma A,K'}\lar\&\hmaps{\Sigma X,K'}\lar
\end{diagram}
Otteniamo così la successione esatta lunga
\begin{diagram}[column sep=small]
\ldots\&\hmaps{X/A,K_{n+1}}\lar\&\hmaps{A,K_n}\lar\&\hmaps{X,K_n}\lar\&\hmaps{X/A,K_n}\lar\&\ldots\lar
\end{diagram}
naturale in $(X,A)$.
\end{onlyenv}
\end{frame*}