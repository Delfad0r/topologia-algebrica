\section*{Aggiunzione \texorpdfstring{$\Sigma\dashv\Omega$}{Sigma-Omega}}
\subsection*{Categorie \texorpdfstring{\Top0}{Top.} e \texorpdfstring{\CW0}{CW.}}
\begin{frame*}
Lavoreremo nella categoria \Top0 e nella sua sottocategoria \CW0.
\pause
\begin{itemize}[<+->]
\item Gli oggetti sono gli spazi topologici (o CW-complessi) puntati $(X,x_0)$.
\item I morfismi sono
\[
\Hom((X,x_0),(Y,y_0))=\tikzmarknode{mark}{\hmaps{X,Y}},
\]
%funzioni continue $\map{f}{(X,x_0)}{(Y,y_0)}$ a meno di omotopia che fissa il punto base.
\begin{center}
\vspace{.5cm}
\begin{tikzpicture}[every node/.style={overlay text}]
\node[inner xsep=0] (a) {$\displaystyle\Big\{f:{(X,x_0)}\to{(Y,y_0)}\tikzmarknode{mark2}{\Big\}}\Big/$};
\node[inner xsep=0,anchor=west,font=\scriptsize,align=left,yshift=-3pt] at (a.east) {omotopia che fissa\\il punto base};
\end{tikzpicture}
\tikz[overlay,remember picture]\draw[overlay arrow,shorten >=2pt] (mark.south) to[out=-100,in=80] (mark2.north);
\end{center}
\item La composizione
\[
\map{{}\circ{}}{\hmaps{Y,Z}\times\hmaps{X,Y}}{\hmaps{X,Z}}
\]
è ben definita.
\end{itemize}
\begin{example}<+->
Abbiamo l'uguaglianza (per ora solo insiemistica)
\[
\pi_n(X)=\hmaps{S^n,X}.
\]
\end{example}
\end{frame*}

\subsection*{Sospensione}
\begin{frame*}
\begin{itemize}
\item La sospensione $SX$ di uno spazio topologico $X$ è
\begin{align*}
SX=X\times[0,1]/\!\sim,&&\text{$X\times\{0\}$ e $X\times\{1\}$ collassati a due punti.}
\end{align*}
\item<3-> Per uno spazio puntato $(X,x_0)$, è conveniente considerare la sospensione ridotta
\begin{align*}
\Sigma X=SX/\{x_0\}\times[0,1],&&\text{con punto base $[x_0]$.}
\end{align*}
\item<4-> Ogni $f\in\hmaps{X,Y}$ induce $\Sigma f\in\hmaps{\Sigma X,\Sigma Y}$:
\[
\Sigma f(x,t)=(f(x),t).
\]
\only<5->{\item La sospensione ridotta è dunque un funtore
\[
\map{\Sigma}{\Top0}{\Top0}
\]
che si restringe a
\[
\map{\Sigma}{\CW0}{\CW0}.
\]}
\end{itemize}
\begin{onlyenv}<-4>
\vspace{-.6cm}
\begin{center}
\begin{tikzpicture}
\def\h{1.2}
\def\st{1/20}
\def\base{circle (1 and .25)}
\def\poslist{.5/.5,0/1,-.5/.5}
% <1>
\begin{onlyenv}<1>
\node[left,inner sep=10pt] at (-1,0) {$X\times[0,1]$};
\foreach \p/\s in {1/1,.5/.5,0/1,-.5/.5,-1/1} {\draw[yshift=\h*\p cm,opacity=\s] \base;}
\path[decorate,decoration={markings,mark=between positions .5 and 1 step \st with {\coordinate (a-\pgfkeysvalueof{/pgf/decoration/mark info/sequence number});}}] \base;
\foreach \i in {1,...,11} {\draw[shift=(a-\i),opacity=.5] (0,-\h) -- (0,\h);}
\end{onlyenv}
\begin{onlyenv}<2->
\node[left,inner sep=10pt] at (-1,0) {$\alt<2>{SX}{\Sigma X}$};
\foreach \p/\s in \poslist {\draw[yshift=\h*\p cm,opacity=\s,scale=\s] \base;}
\path[suspension={apex={(0,\h)},apex style={alt={<3->{fill=blue}{fill=black}}},step=\st,between=.5 and 1,highlight=at -.15 with {thick,blue,onslide=<3->}}]\base;
\end{onlyenv}
\begin{onlyenv}<4->
\draw (1.5,0) edge[->,"\scriptsize$\Sigma f$"] (3,0);
\begin{scope}[xshift=4.5cm]
\node[right,inner sep=10pt] at (1,0) {$\Sigma Y$};
\foreach \p/\s in \poslist {\draw[yshift=\h*\p cm,opacity=\s,scale=\s] \base;}
\path[suspension={apex={(0,\h)},apex style={alt={<3->{fill=blue}{fill=black}}},step=\st,between=.5 and 1,highlight=at -.15 with {thick,blue,onslide=<3->}}]\base;
\end{scope}
\end{onlyenv}
\end{tikzpicture}
\end{center}
\end{onlyenv}
\end{frame*}

\begin{frame*}
\begin{overlayarea}{\textwidth}{.825\textheight}
Dati due spazi topologici puntati $X,Y$, esiste una struttura di gruppo canonica su $\hmaps{\Sigma X,Y}$.
\begin{itemize}
\item<2-> Per $f,g\in\hmaps{\Sigma X,Y}$, il prodotto $f\cdot g$ è così definito:\only<3->{
\[
f\cdot g(x,t)=\begin{dcases}f(x,2t)&t\le\frac{1}{2},\\g(x,2t-1)&t\ge\frac{1}{2}.\end{dcases}
\]
\item<4-> Si verifica facilmente che questa operazione definisce una struttura di gruppo, avente la funzione costante come elemento neutro.
\item<5-> Per ogni $\map{f}{Y}{Z}$, la composizione
\[
\map{f\circ-{}}{\hmaps{\Sigma X,Y}}{\hmaps{\Sigma X,Z}}
\]
 è un omomorfismo di gruppi.
 \item<6-> Di conseguenza, otteniamo il funtore
 \[
 \map{\hmaps{\Sigma X,-}}{\Top0}{\Grp}.
 \]
 }
\end{itemize}
\begin{onlyenv}<2>
\begin{center}
\begin{tikzpicture}
\def\h{1.2}
\def\st{1/20}
\def\base{circle (1. and .25)}
\def\poslist{.5/.5,0/1,-.5/.5}
\matrix[ampersand replacement=\&,row 1/.style={anchor=base,inner xsep=5pt},row sep=.5cm,column sep={4cm,between origins}]{
\node (SX) {$\Sigma X$};\&\node (SXvSX) {$\Sigma X\vee\Sigma X$};\&\node (Y) {$Y$};\\
\foreach \p/\s in {.5/.5,0/1,-.5/.5} {\draw[yshift=\h*\p cm,opacity=\s,scale=\s] \base;}\path[suspension={apex={(0,\h)},step=1/20,between=.5 and 1}] \base;\coordinate (a0) at(1,0);\&\foreach \p[count=\i] in {.5,-.5} {\path[yshift=\h*\p cm,scale=.5,preaction={draw=black,opacity=.5},suspension={apex={(0,\h/2)},apex style={scale=2},step=1/20,between=.5 and 1}] \base;\coordinate (a\i) at (.5,\h*\p);}\&\draw[rounded corners,decorate,decoration={random steps,segment length=.4cm,amplitude=.1cm,pre=lineto,pre length=.25cm,post=lineto,post length=.25cm}] circle (.6);\coordinate (a3);\\
};
\draw[->] (SX) edge (SXvSX) (SXvSX) edge["\scriptsize$f\vee g$"] (Y);
\draw[->,shorten <=.5cm,shorten >=.5cm] ++(a0) to +(2.5,0);
\draw[shorten <=.5cm,shorten >=1.1cm] (a1) edge[->,bend left=15,"\scriptsize$f$" pos=.35] (a3) (a2) edge[->,bend right=15,"\scriptsize$g$"' pos=.35] (a3);
\end{tikzpicture}
\end{center}
\end{onlyenv}
\end{overlayarea}
\end{frame*}

\subsection*{Spazio di lacci}
\begin{frame*}
\begin{itemize}[<+->]
\item Dato uno spazio topologico puntato $(X,x_0)$, definiamo lo spazio di lacci
\[
\Omega X=\left\{g\in X^{[0,1]}:g(0)=g(1)=x_0\right\}
\]
con punto base il cammino costante in $x_0$. La topologia è quella compatta-aperta.
\item Ogni $f\in\hmaps{X,Y}$ induce $\Omega f\in\hmaps{\Omega X,\Omega Y}$ per composizione:
\[
\Omega f(g)=f\circ g.
\]
 \item Lo spazio di lacci è dunque un funtore
\[
\map{\Omega}{\Top0}{\Top0}.
\]
\end{itemize}
\end{frame*}

\begin{frame*}
In modo duale a quanto accade per la sospensione, dati due spazi topologici puntati $X,Y$, esiste una struttura di gruppo canonica su $\hmaps{X,\Omega Y}$.
\pause
\begin{itemize}[<+->]
\item Per $f,g\in\hmaps{X,\Omega Y}$, il prodotto $f\cdot g$ è così definito:
\[
f\cdot g(x)(t)=\begin{dcases}f(x)(2t)&t\le\frac{1}{2},\\g(x)(2t-1)&t\ge\frac{1}{2}.\end{dcases}
\]
\item Si verifica facilmente che questa operazione definisce una struttura di gruppo, avente la funzione costante come elemento neutro.
\item Per ogni $\map{f}{X}{Z}$, la composizione
\[
\map{-\circ f}{\hmaps{Z,\Omega Y}}{\hmaps{X,\Omega Y}}
\]
 è un omomorfismo di gruppi.
 \item Di conseguenza, otteniamo il funtore
 \[
 \map{\hmaps{-,\Omega Y}}{\Top0}{\Grp^{\text{op}}}.
 \]
\end{itemize}
\end{frame*}

\subsection*{Osservazioni}
\begin{frame*}
\begin{itemize}[<+->]
\item Con la struttura di gruppo che abbiamo definito, l'uguaglianza
\[
\pi_n(X)=\hmaps{S^n,X}=\hmaps{\Sigma S^{n-1},X}
\]
è un isomorfismo di gruppi.
\item Le due strutture di gruppo su  $\hmaps{\Sigma X,\Omega Y}$ coincidono.
\item $\Omega^2Y$ può essere interpretato come lo spazio delle funzioni
\[
\umap{[0,1]\times[0,1]}{Y}
\]
che assumono il valore $y_0$ sul bordo. Con lo stesso ragionamento usato per il $\pi_2$, si dimostra che $\hmaps{X,\Omega^2Y}$ è abeliano.
\end{itemize}
\end{frame*}

\subsection*{Aggiunzione}
\begin{frame*}
\begin{proposition}
I funtori $\Sigma$ e $\Omega$ sono aggiunti. In altre parole, esiste una biiezione
\[
\umap[\iso]{\hmaps{\Sigma X,Y}}{\hmaps{X,\Omega Y}}
\]
naturale in $X$ e $Y$. Inoltre queste biiezioni sono isomorfismi di gruppi.
\end{proposition}
\pause
\begin{remarks}
\begin{itemize}[<+->]
\item Considerando $S^{n-1}$, otteniamo che
\[
\pi_n(X)=\hmaps{S^n,X}=\hmaps{\Sigma S^{n-1},X}\iso\hmaps{S^{n-1},\Omega X}=\pi_{n-1}(\Omega X).
\]
\item In particolare, se $X$ è un $K(G,n)$, allora $\Omega X$ è un $K(G,n-1)$.
\item Per ogni $n\ge 1$, sia $K_n$ un CW-complesso che è anche un $K(G,n)$; per unicità dei $K(G,n)$, abbiamo equivalenze omotopiche deboli
\[
\map{\theta_n}{K_n}{\Omega K_{n+1}}.
\]
\end{itemize}
\end{remarks}
\end{frame*}