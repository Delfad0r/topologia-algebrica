\section*{Aggiunzione \texorpdfstring{$\Sigma\dashv\Omega$}{Sigma-Omega}}
\subsection*{Categorie \texorpdfstring{\Top0}{Top.} e \texorpdfstring{\CW0}{CW.}}
\begin{frame*}
Lavoreremo nella categoria \Top0 e nella sua sottocategoria \CW0.
\begin{itemize}
\item Gli oggetti sono gli spazi topologici (o CW-complessi) puntati $(X,x_0)$.
\item I morfismi sono
\[
\Hom((X,x_0),(Y,y_0))=\hmaps{X,Y},
\]
funzioni continue $\map{f}{(X,x_0)}{(Y,y_0)}$ a meno di omotopia che fissa il punto base.
\item La composizione
\[
\map{{}\circ{}}{\hmaps{Y,Z}\times\hmaps{X,Y}}{\hmaps{X,Z}}
\]
è ben definita.
\end{itemize}
\begin{example}
Abbiamo l'uguaglianza (per ora solo insiemistica)
\[
\pi_n(X)=\hmaps{S^n,X}.
\]
\end{example}
\end{frame*}

\begin{comment}

\subsection*{Equivalenze omotopiche deboli}
\begin{frame*}
\begin{definition}
Una funzione $\map{f}{X}{Y}$ si dice \emph{equivalenza omotopica debole} se per ogni $n\ge 0$  la mappa indotta
\[
\map{f_*}{\pi_n(X)}{\pi_n(Y)}
\]
è un isomorfismo.
\end{definition}
\begin{proposition}
Siano $X$ un CW-complesso, $\map{f}{Y}{Z}$ un'equivalenza omotopica debole di spazi topologici. Allora
\[
\map{f\circ-}{\hmaps{X,Y}}{\hmaps{X,Z}}
\]
è una biiezione.
\end{proposition}
\begin{theorem}[Approssimazione CW]
Per ogni spazio topologico $X$ esistono un CW-complesso $Z$ e un'equivalenza omotopica debole
$\map{f}{Z}{X}$.
\end{theorem}
\end{frame*}
\end{comment}

\subsection*{Sospensione}
\begin{frame*}
\begin{itemize}
\item La sospensione $SX$ di uno spazio topologico $X$ è
\begin{align*}
SX=X\times[0,1]/\!\sim,&&\text{$X\times\{0\}$ e $X\times\{1\}$ collassati a due punti.}
\end{align*}
\item Per uno spazio puntato $(X,x_0)$, è conveniente considerare la sospensione ridotta
\begin{align*}
\Sigma X=SX/\{x_0\}\times[0,1],&&\text{con punto base $[x_0]$.}
\end{align*}
\item Ogni $f\in\hmaps{X,Y}$ induce $\Sigma f\in\hmaps{\Sigma X,\Sigma Y}$:
\[
\Sigma f(x,t)=(f(x),t).
\]
\item La sospensione ridotta è dunque un funtore
\[
\map{\Sigma}{\Top0}{\Top0}
\]
che si restringe a
\[
\map{\Sigma}{\CW0}{\CW0}.
\]
\end{itemize}
\end{frame*}

\begin{frame*}
Dati due spazi topologici puntati $X,Y$, esiste una struttura di gruppo canonica su $\hmaps{\Sigma X,Y}$.
\begin{itemize}
\item Per $f,g\in\hmaps{\Sigma X,Y}$, il prodotto $f\cdot g$ è così definito:
\[
f\cdot g(x,t)=\begin{dcases}f(x,2t)&t\le\frac{1}{2},\\g(x,2t-1)&t\ge\frac{1}{2}.\end{dcases}
\]
\item Si verifica facilmente che questa operazione definisce una struttura di gruppo, avente la funzione costante come elemento neutro.
\item Per ogni $\map{f}{Y}{Z}$, la composizione
\[
\map{f\circ-{}}{\hmaps{\Sigma X,Y}}{\hmaps{\Sigma X,Z}}
\]
 è un omomorfismo di gruppi.
 \item Di conseguenza, otteniamo il funtore
 \[
 \map{\hmaps{\Sigma X,-}}{\Top0}{\Grp}.
 \]
\end{itemize}
\end{frame*}

\subsection*{Spazio di lacci}
\begin{frame*}
\begin{itemize}
\item Dato uno spazio topologico puntato $(X,x_0)$, definiamo lo spazio di lacci
\[
\Omega X=\left\{g\in X^{[0,1]}:g(0)=g(1)=x_0\right\}
\]
con punto base il cammino costante in $x_0$. La topologia è quella compatta-aperta.
\item Ogni $f\in\hmaps{X,Y}$ induce $\Omega f\in\hmaps{\Omega X,\Omega Y}$ per composizione:
\[
\Omega f(g)=f\circ g.
\]
 \item Lo spazio di lacci è dunque un funtore
\[
\map{\Omega}{\Top0}{\Top0}.
\]
\end{itemize}
\end{frame*}

\begin{frame*}
In modo duale a quanto accade per la sospensione, dati due spazi topologici puntati $X,Y$, esiste una struttura di gruppo canonica su $\hmaps{X,\Omega Y}$.
\begin{itemize}
\item Per $f,g\in\hmaps{X,\Omega Y}$, il prodotto $f\cdot g$ è così definito:
\[
f\cdot g(x)(t)=\begin{dcases}f(x)(2t)&t\le\frac{1}{2},\\g(x)(2t-1)&t\ge\frac{1}{2}.\end{dcases}
\]
\item Si verifica facilmente che questa operazione definisce una struttura di gruppo, avente la funzione costante come elemento neutro.
\item Per ogni $\map{f}{X}{Z}$, la composizione
\[
\map{-\circ f}{\hmaps{Z,\Omega Y}}{\hmaps{X,\Omega Y}}
\]
 è un omomorfismo di gruppi.
 \item Di conseguenza, otteniamo il funtore
 \[
 \map{\hmaps{-,\Omega Y}}{\Top0}{\Grp^{\text{op}}}.
 \]
\end{itemize}
\end{frame*}

\subsection*{Osservazioni}
\begin{frame*}
\begin{itemize}
\item Con la struttura di gruppo che abbiamo definito, l'uguaglianza
\[
\pi_n(X)=\hmaps{S^n,X}=\hmaps{\Sigma S^{n-1},X}
\]
è un isomorfismo di gruppi.
\item Le due strutture di gruppo su  $\hmaps{\Sigma X,\Omega Y}$ coincidono.
\item $\Omega^2Y$ può essere interpretato come lo spazio delle funzioni
\[
\umap{[0,1]\times[0,1]}{Y}
\]
che assumono il valore $y_0$ sul bordo. Con lo stesso ragionamento usato per il $\pi_2$, si dimostra che $\hmaps{X,\Omega^2Y}$ è abeliano.
\end{itemize}
\end{frame*}

\subsection*{Aggiunzione}
\begin{frame*}
\begin{proposition}
I funtori $\Sigma$ e $\Omega$ sono aggiunti. In altre parole, esiste una biiezione
\[
\umap[\iso]{\hmaps{\Sigma X,Y}}{\hmaps{X,\Omega Y}}
\]
naturale in $X$ e $Y$. Inoltre queste biiezioni sono isomorfismi di gruppi.
\end{proposition}
\begin{remarks}
\begin{itemize}
\item Considerando $S^{n-1}$, otteniamo che
\[
\pi_n(X)=\hmaps{S^n,X}=\hmaps{\Sigma S^{n-1},X}\iso\hmaps{S^{n-1},\Omega X}=\pi_{n-1}(\Omega X).
\]
\item In particolare, se $X$ è un $K(G,n)$, allora $\Omega X$ è un $K(G,n-1)$.
\item Per ogni $n\ge 1$, sia $K_n$ un CW-complesso che è anche un $K(G,n)$; per unicità dei $K(G,n)$, abbiamo equivalenze omotopiche deboli
\[
\map{\theta_n}{K_n}{\Omega K_{n+1}}.
\]
\end{itemize}
\end{remarks}
\end{frame*}