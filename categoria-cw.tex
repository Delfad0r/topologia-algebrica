\section*{Categoria \texorpdfstring{\CW0}{CW.}}
\subsection*{Definizione}
\begin{frame*}
Lavoreremo prevalentemente nella categoria \CW0.
\begin{itemize}
\item Gli oggetti sono i CW-complessi puntati $(X,x_0)$.
\item I morfismi sono
\[
\Hom((X,x_0),(Y,y_0))=\hmaps{X,Y},
\]
funzioni continue $\map{f}{(X,x_0)}{(Y,y_0)}$ a meno di omotopia che fissa il punto base.
\item La composizione
\[
\map{{}\circ{}}{\hmaps{Y,Z}\times\hmaps{X,Y}}{\hmaps{X,Z}}
\]
è ben definita.
\end{itemize}
\begin{example}
Abbiamo l'uguaglianza (per ora solo insiemistica)
\[
\pi_n(X)=\hmaps{S^n,X}.
\]
\end{example}
\end{frame*}

\subsection*{Equivalenze omotopiche deboli}
\begin{frame*}
\begin{definition}
Una funzione $\map{f}{X}{Y}$ si dice \emph{equivalenza omotopica debole} se per ogni $n\ge 0$  la mappa indotta
\[
\map{f_*}{\pi_n(X)}{\pi_n(Y)}
\]
è un isomorfismo.
\end{definition}
\begin{proposition}
Siano $X$ un CW-complesso, $\map{f}{Y}{Z}$ un'equivalenza omotopica debole di spazi topologici. Allora
\[
\map{f\circ-}{\hmaps{X,Y}}{\hmaps{X,Z}}
\]
è una biiezione.
\end{proposition}
\begin{theorem}[Approssimazione CW]
Per ogni spazio topologico $X$ esistono un CW-complesso $Z$ e un'equivalenza omotopica debole
$\map{f}{Z}{X}$.
\end{theorem}
\end{frame*}

\subsection*{Funtore \texorpdfstring{$\Sigma$}{Sigma}}
\begin{frame*}
\begin{itemize}
\item La sospensione $SX$ di uno spazio topologico $X$ è
\begin{align*}
SX=X\times[0,1]/\sim,&&\text{$X\times\{0\}$ e $X\times\{1\}$ collassati a due punti.}
\end{align*}
\item Per un CW-complesso puntato $(X,x_0)$, è conveniente considerare la sospensione ridotta
\begin{align*}
\Sigma X=SX/\{x_0\}\times[0,1],&&\text{con punto base $[x_0]$.}
\end{align*}
\item Ogni $f\in\hmaps{X,Y}$ induce $\Sigma f\in\hmaps{\Sigma X,\Sigma Y}$; la sospensione ridotta è dunque un funtore
\[
\map{\Sigma}{\CW0}{\CW0}.
\]
\end{itemize}
\end{frame*}

\subsection*{Funtore \texorpdfstring{$\Omega$}{Omega}}
\begin{frame*}
content...
\end{frame*}