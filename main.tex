\documentclass[9pt]{beamer}
\usepackage{mystyle}

\begin{document}

\section*{Costruzione omotopica della coomologia}
\subsection*{Enunciato}
\begin{frame*}
\begin{theorem}
Siano $X$ un CW-complesso, $G$ un gruppo abeliano, $n\ge 0$ un intero. Allora esiste una biiezione
\[
\map[\iso]{T}{\hmaps{X,K(G,n)}}{\til{H}^n(X;G)}
\]
naturale in $X$.
\end{theorem}
\begin{itemize}
\item Esiste una struttura canonica di gruppo abeliano su $\hmaps{X,K(G,n)}$ che rende $T$ un isomorfismo di gruppi.
\item $T$ è della forma
\[
T([f])=f^*(\alpha)
\]
dove $\alpha$ è la ``classe fondamentale'' di $H^n(K(G,n);G)$.
\end{itemize}
\end{frame*}

\subsection*{Strategia dimostrativa}
\begin{frame*}
\begin{enumerate}
\addtolength\itemsep{1em}
\item Struttura di gruppo su $\hmaps{X,K(G,n)}$ $\Longrightarrow$ definizione di $\Omega$-spettro.
\item Per ogni $\Omega$-spettro $\{K_n\}$, la famiglia di funtori $h^n=\hmaps{-,K_n}$ è una teoria coomologica ridotta sulla categoria dei CW-complessi puntati.
\item Se una teoria coomologica ridotta $h^*$ soddisfa $h^n(S^0)=0$ per $n\neq 0$, allora esistono isomorfismi naturali $h^n(X)\iso\til{H}^n(X;h^0(S^0))$.
\end{enumerate}
\end{frame*}

\section*{Aggiunzione \texorpdfstring{$\Sigma\dashv\Omega$}{Sigma-Omega}}
\subsection*{Categorie \texorpdfstring{\Top0}{Top.} e \texorpdfstring{\CW0}{CW.}}
\begin{frame*}
Lavoreremo nella categoria \Top0 e nella sua sottocategoria \CW0.
\begin{itemize}
\item Gli oggetti sono gli spazi topologici (o CW-complessi) puntati $(X,x_0)$.
\item I morfismi sono
\[
\Hom((X,x_0),(Y,y_0))=\hmaps{X,Y},
\]
funzioni continue $\map{f}{(X,x_0)}{(Y,y_0)}$ a meno di omotopia che fissa il punto base.
\item La composizione
\[
\map{{}\circ{}}{\hmaps{Y,Z}\times\hmaps{X,Y}}{\hmaps{X,Z}}
\]
è ben definita.
\end{itemize}
\begin{example}
Abbiamo l'uguaglianza (per ora solo insiemistica)
\[
\pi_n(X)=\hmaps{S^n,X}.
\]
\end{example}
\end{frame*}

\begin{comment}

\subsection*{Equivalenze omotopiche deboli}
\begin{frame*}
\begin{definition}
Una funzione $\map{f}{X}{Y}$ si dice \emph{equivalenza omotopica debole} se per ogni $n\ge 0$  la mappa indotta
\[
\map{f_*}{\pi_n(X)}{\pi_n(Y)}
\]
è un isomorfismo.
\end{definition}
\begin{proposition}
Siano $X$ un CW-complesso, $\map{f}{Y}{Z}$ un'equivalenza omotopica debole di spazi topologici. Allora
\[
\map{f\circ-}{\hmaps{X,Y}}{\hmaps{X,Z}}
\]
è una biiezione.
\end{proposition}
\begin{theorem}[Approssimazione CW]
Per ogni spazio topologico $X$ esistono un CW-complesso $Z$ e un'equivalenza omotopica debole
$\map{f}{Z}{X}$.
\end{theorem}
\end{frame*}
\end{comment}

\subsection*{Sospensione}
\begin{frame*}
\begin{itemize}
\item La sospensione $SX$ di uno spazio topologico $X$ è
\begin{align*}
SX=X\times[0,1]/\!\sim,&&\text{$X\times\{0\}$ e $X\times\{1\}$ collassati a due punti.}
\end{align*}
\item<3-> Per uno spazio puntato $(X,x_0)$, è conveniente considerare la sospensione ridotta
\begin{align*}
\Sigma X=SX/\{x_0\}\times[0,1],&&\text{con punto base $[x_0]$.}
\end{align*}
\item<4-> Ogni $f\in\hmaps{X,Y}$ induce $\Sigma f\in\hmaps{\Sigma X,\Sigma Y}$:
\[
\Sigma f(x,t)=(f(x),t).
\]
\only<5->{\item La sospensione ridotta è dunque un funtore
\[
\map{\Sigma}{\Top0}{\Top0}
\]
che si restringe a
\[
\map{\Sigma}{\CW0}{\CW0}.
\]}
\end{itemize}
\begin{onlyenv}<-4>
\vspace{-.6cm}
\begin{center}
\begin{tikzpicture}[framed]
\def\h{1.2}
\def\st{1/20}
\def\base{circle (1 and .2)}
\def\poslist{.5/.5,0/1,-.5/.5}
% <1>
\begin{onlyenv}<1>
\node[left,inner sep=10pt] at (-1,0) {$X\times[0,1]$};
\foreach \p/\s in {1/1,.5/.5,0/1,-.5/.5,-1/1} {\draw[yshift=\h*\p cm,opacity=\s] \base;}
\path[decorate,decoration={markings,mark=between positions .5 and 1 step \st with {\coordinate (a-\pgfkeysvalueof{/pgf/decoration/mark info/sequence number});}}] \base;
\foreach \i in {1,...,11} {\draw[shift=(a-\i),opacity=.5] (0,-\h) -- (0,\h);}
\end{onlyenv}
\begin{onlyenv}<2->
\node[left,inner sep=10pt] at (-1,0) {$\alt<2>{SX}{\Sigma X}$};
\foreach \p/\s in \poslist {\draw[yshift=\h*\p cm,opacity=\s,scale=\s] \base;}
\path[suspension={apex={(0,\h)},apex style={alt={<3->{fill=blue}{fill=black}}},step=\st,between=.5 and 1,highlight=at -.15 with {thick,blue,onslide=<3->}}]\base;
\end{onlyenv}
\begin{onlyenv}<4->
\draw (1.5,0) edge[->,"$\Sigma f$"] (3,0);
\begin{scope}[xshift=4.5cm]
\node[right,inner sep=10pt] at (1,0) {$\Sigma Y$};
\foreach \p/\s in \poslist {\draw[yshift=\h*\p cm,opacity=\s,scale=\s] \base;}
\path[suspension={apex={(0,\h)},apex style={alt={<3->{fill=blue}{fill=black}}},step=\st,between=.5 and 1,highlight=at -.15 with {thick,blue,onslide=<3->}}]\base;
\end{scope}
\end{onlyenv}
\end{tikzpicture}
\end{center}
\end{onlyenv}
\end{frame*}

\begin{frame*}
\begin{overlayarea}{\textwidth}{.825\textheight}
Dati due spazi topologici puntati $X,Y$, esiste una struttura di gruppo canonica su $\hmaps{\Sigma X,Y}$.
\begin{itemize}
\item<2-> Per $f,g\in\hmaps{\Sigma X,Y}$, il prodotto $f\cdot g$ è così definito:\visible<3->{
\[
f\cdot g(x,t)=\begin{dcases}f(x,2t)&t\le\frac{1}{2},\\g(x,2t-1)&t\ge\frac{1}{2}.\end{dcases}
\]}\only<4->{
\item Si verifica facilmente che questa operazione definisce una struttura di gruppo, avente la funzione costante come elemento neutro.
\item<5-> Per ogni $\map{f}{Y}{Z}$, la composizione
\[
\map{f\circ-{}}{\hmaps{\Sigma X,Y}}{\hmaps{\Sigma X,Z}}
\]
 è un omomorfismo di gruppi.
 \item<6-> Di conseguenza, otteniamo il funtore
 \[
 \map{\hmaps{\Sigma X,-}}{\Top0}{\Grp}.
 \]
 }
\end{itemize}
\begin{onlyenv}<2>
\begin{center}
\begin{tikzpicture}[framed]
\draw circle (1);
\end{tikzpicture}
\end{center}
\end{onlyenv}
\end{overlayarea}
\end{frame*}

\subsection*{Spazio di lacci}
\begin{frame*}
\begin{itemize}
\item Dato uno spazio topologico puntato $(X,x_0)$, definiamo lo spazio di lacci
\[
\Omega X=\left\{g\in X^{[0,1]}:g(0)=g(1)=x_0\right\}
\]
con punto base il cammino costante in $x_0$. La topologia è quella compatta-aperta.
\item Ogni $f\in\hmaps{X,Y}$ induce $\Omega f\in\hmaps{\Omega X,\Omega Y}$ per composizione:
\[
\Omega f(g)=f\circ g.
\]
 \item Lo spazio di lacci è dunque un funtore
\[
\map{\Omega}{\Top0}{\Top0}.
\]
\end{itemize}
\end{frame*}

\begin{frame*}
In modo duale a quanto accade per la sospensione, dati due spazi topologici puntati $X,Y$, esiste una struttura di gruppo canonica su $\hmaps{X,\Omega Y}$.
\begin{itemize}
\item Per $f,g\in\hmaps{X,\Omega Y}$, il prodotto $f\cdot g$ è così definito:
\[
f\cdot g(x)(t)=\begin{dcases}f(x)(2t)&t\le\frac{1}{2},\\g(x)(2t-1)&t\ge\frac{1}{2}.\end{dcases}
\]
\item Si verifica facilmente che questa operazione definisce una struttura di gruppo, avente la funzione costante come elemento neutro.
\item Per ogni $\map{f}{X}{Z}$, la composizione
\[
\map{-\circ f}{\hmaps{Z,\Omega Y}}{\hmaps{X,\Omega Y}}
\]
 è un omomorfismo di gruppi.
 \item Di conseguenza, otteniamo il funtore
 \[
 \map{\hmaps{-,\Omega Y}}{\Top0}{\Grp^{\text{op}}}.
 \]
\end{itemize}
\end{frame*}

\subsection*{Osservazioni}
\begin{frame*}
\begin{itemize}
\item Con la struttura di gruppo che abbiamo definito, l'uguaglianza
\[
\pi_n(X)=\hmaps{S^n,X}=\hmaps{\Sigma S^{n-1},X}
\]
è un isomorfismo di gruppi.
\item Le due strutture di gruppo su  $\hmaps{\Sigma X,\Omega Y}$ coincidono.
\item $\Omega^2Y$ può essere interpretato come lo spazio delle funzioni
\[
\umap{[0,1]\times[0,1]}{Y}
\]
che assumono il valore $y_0$ sul bordo. Con lo stesso ragionamento usato per il $\pi_2$, si dimostra che $\hmaps{X,\Omega^2Y}$ è abeliano.
\end{itemize}
\end{frame*}

\subsection*{Aggiunzione}
\begin{frame*}
\begin{proposition}
I funtori $\Sigma$ e $\Omega$ sono aggiunti. In altre parole, esiste una biiezione
\[
\umap[\iso]{\hmaps{\Sigma X,Y}}{\hmaps{X,\Omega Y}}
\]
naturale in $X$ e $Y$. Inoltre queste biiezioni sono isomorfismi di gruppi.
\end{proposition}
\begin{remarks}
\begin{itemize}
\item Considerando $S^{n-1}$, otteniamo che
\[
\pi_n(X)=\hmaps{S^n,X}=\hmaps{\Sigma S^{n-1},X}\iso\hmaps{S^{n-1},\Omega X}=\pi_{n-1}(\Omega X).
\]
\item In particolare, se $X$ è un $K(G,n)$, allora $\Omega X$ è un $K(G,n-1)$.
\item Per ogni $n\ge 1$, sia $K_n$ un CW-complesso che è anche un $K(G,n)$; per unicità dei $K(G,n)$, abbiamo equivalenze omotopiche deboli
\[
\map{\theta_n}{K_n}{\Omega K_{n+1}}.
\]
\end{itemize}
\end{remarks}
\end{frame*}
\section*{\texorpdfstring{$\Omega$}{Omega}-spettri e coomologia}

\subsection*{Definizione}
\begin{frame*}
\begin{definition}
Un \emph{$\Omega$-spettro} è una famiglia di CW-complessi $\{K_n\}_{n\ge 1}$ dotata di equivalenze omotopiche deboli
\[
\map{\theta_n}{K_n}{\Omega K_{n+1}}.
\]
\end{definition}
\pause
\begin{itemize}
\item<+-> È possibile estendere la famiglia anche a indici $n\le 0$: è sufficiente considerare come $K_{n-1}$ un'approssimazione CW di $\Omega K_n$.
\end{itemize}
\begin{proposition}<+->
Siano $\map{f}{Y}{Z}$ un'equivalenza omotopica debole, $X$ un CW-complesso. Allora la composizione
\[
\map[\iso]{f\circ-{}}{\hmaps{X,Y}}{\hmaps{X,Z}}
\]
è una biiezione.
\end{proposition}
\end{frame*}

\subsection*{Struttura di gruppo}
\begin{frame*}
D'ora in poi, tutti gli spazi di cui parleremo saranno CW-complessi puntati.\pause
\begin{itemize}[<+->]
\item Possiamo dare a $\hmaps{X,K_n}$ una struttura di gruppo imponendo che
\[
\map[\iso]{\theta_n\circ-{}}{\hmaps{X,K_n}}{\hmaps{X,\Omega K_{n+1}}}
\]
sia un isomorfismo di gruppi.
\item In questo modo, $\hmaps{X,K_n}$ risulta essere un gruppo abeliano:
\begin{diagram}[column sep=huge]
\&\hmaps{\Sigma X,K_{n+1}}\rar["\iso"']{\theta_{n+1}\circ-}\dar{\iso}\&\hmaps{\Sigma X,\Omega K_{n+2}}\dar{\iso}\\
\hmaps{X,K_n}\rar["\iso"']{\theta_n\circ-}\&\hmaps{X,\Omega K_{n+1}}\rar{\Omega\theta_{n+1}\circ-}\&\hmaps{X,\Omega^2 K_{n+2}}\nospacepunct{.}
\end{diagram}
\item Di conseguenza, per ogni $n\in\ZZ$ otteniamo il funtore
\[
\map{\hmaps{-,K_n}}{\CW0}{\Ab^{\text{op}}}.
\]
\end{itemize}
\end{frame*}

\subsection*{Teorie coomologiche}
\begin{frame*}
\begin{theorem}
Sia $\{K_n\}_{n\in\ZZ}$ un $\Omega$-spettro. Allora i funtori $h^n=\hmaps{-,K_n}$ definiscono una teoria coomologica ridotta sulla categoria \CW0.
\end{theorem}\pause
\begin{reminder}<+->
Una \emph{teoria coomologica ridotta} sulla categoria \CW0 è una famiglia di funtori
\begin{align*}
\map{h^n}{\CW0}{\Ab^{\text{op}}},&&n\in\ZZ
\end{align*}
che soddisfa i seguenti assiomi.
\begin{enumerate}[<+->]
\item\label{cohomology-axiom-les} Per ogni coppia $(X,A)$ esiste una successione esatta lunga
\begin{diagram}[column sep=small]
\ldots\rar{\delta}\&h^n(X/A)\rar{q^*}\&h^n(X)\rar{i^*}\&h^n(A)\rar{\delta}\&h^{n+1}(X/A)\rar{q^*}\&\ldots
\end{diagram}
naturale in $(X,A)$.
\item\label{cohomology-axiom-wedge} Per ogni famiglia $\{X_\alpha\}_{\alpha}$, le inclusioni inducono un isomorfismo
\[
\umap[\iso]{h^n\left(\textstyle\bigvee_\alpha X_\alpha\right)}{\textstyle\prod_\alpha h^n(X_\alpha)}.
\]
\end{enumerate}
\end{reminder}
\end{frame*}

\subsection*{Dimostrazione}
\begin{frame*}
\begin{itemize}[<+->]
\item L'assioma \ref{cohomology-axiom-wedge}. è soddisfatto: nella categoria \CW0 dare un morfismo
\[
\umap{{\textstyle\bigvee_\alpha X_\alpha}}{K_n}
\]
è equivalente a dare  una collezione di morfismi
\[
\{\umap{X_\alpha}{K_n}\}_\alpha.
\]
\item Per quanto riguarda l'assioma \ref{cohomology-axiom-les}., sia $(X,A)$ una coppia di CW-complessi; vediamo come costruire la successione esatta lunga associata.
\end{itemize}
\end{frame*}

\begin{frame*}
\begin{overlayarea}{\textwidth}{.86\textheight}
\begin{onlyenv}<-11>
\begin{diagram}[column sep=11pt]
\tikzmarknode[alt=<9>{gray}{black}]{mark1}{A}\rar[hook,alt=<9>{gray}{black}]\dar[onslide=<4->,equal,alt=<9>{gray}{black}]\&
|[alt=<9>{gray}{black}]|X\rar[hook,alt=<9>{gray}{black}]\dar[onslide=<4->,equal,alt=<9>{gray}{black}]\&
|[alt=<9>{gray}{black}]|X\cup\cone{A}\rar[onslide=<2->,hook,alt=<9>{gray}{black}]\dar[onslide=<5->,alt=<9>{gray}{black}]{\iso}\&
|[onslide=<2->,alt=<9>{gray}{black}]|(X\cup\cone{A})\cup\cone{X}\rar[hook,onslide=<3->,alt=<9>{gray}{black}]\dar[onslide=<6->,alt=<9>{gray}{black}]{\iso}\&
|[onslide=<3->,alt=<9>{gray}{black}]|\big((X\cup\cone{A})\cup\cone{X}\big)\cup\cone(X\cup\cone{A})\dar[onslide=<7->,alt=<9>{gray}{black}]{\iso}\\
\tikzmarknode{mark2}{\visible<4->{A}}\rar[onslide=<4->,hook]\dar[onslide=<9>]{f}\&
{\visible<4->{X}}\rar[onslide=<8->]\dar[onslide=<9>]{f}\&
{\visible<5->{X/A}}\rar[onslide=<8->]\dar[onslide=<9>]{\bar{f}}\&
{\visible<6->{\Sigma A}}\rar[onslide=<8->,hook]\dar[onslide=<9>]{\Sigma f}\&
{\visible<7->{\Sigma X}}\dar[onslide=<9>]{\Sigma f}\\
{\visible<9>{B}}\rar[onslide=<9>,hook]\&
{\visible<9>{Y}}\rar[onslide=<9>]\&
{\visible<9>{Y/B}}\rar[onslide=<9>]\&
{\visible<9>{\Sigma B}}\rar[onslide=<9>,hook]\&
{\visible<9>{\Sigma Y}}
\end{diagram}
\begin{tikzpicture}[overlay,remember picture,every node/.style={blue,anchor=base,draw,circle,font=\footnotesize,inner sep=1pt}]
\node[left=.3cm of  mark1,onslide=<3->] {\textbf{1}};
\node[left=.3cm of  mark2,onslide=<8->] {\textbf{2}};
\end{tikzpicture}
\end{onlyenv}
\begin{onlyenv}<-9>
\begin{overlayarea}{\textwidth}{2.8cm}
\begin{center}
\begin{tikzpicture}[inner frame ysep=1.5ex,background rectangle/.style={draw=blue!50,fill=blue!8,rounded corners=1ex,drop shadow},framed]
\def\h{.75}
\def\wX{1.5}
\def\wA{.75}
\path (0,-\h) -- (0,\h*1.5);
\matrix[ampersand replacement=\&,A/.style={very thick,line cap=round},column sep=.7cm,cells={}]{
\draw[A] (0,0) -- (\wA,0);\&
\draw[A] (0,0) -- (\wA,0);\draw (0,0) -- (\wX,0);\&
\path[use as bounding box] (0,0) -- (\wX,0);\alt<-4>{
\path[cone={style={blue},step=1.5cm/10,apex style={opacity=0},apex={(\wA/2,\h)}},postaction={draw,A}] (0,0) -- (\wA,0);\draw (0,0) -- (\wX,0);}{\draw (\wA,0) -- (\wX,0);\fill[blue] (\wA,0) circle (1pt);}\&
\path[use as bounding box] (0,0) -- (\wX,0);\temporal<2-5>{}{\path[cone={step=1.5cm/10,apex style={opacity=0},apex={(\wA/2,\h)}},postaction={draw,A}] (0,0) -- (\wA,0);\path[cone={style={blue},step=1.5cm/10,apex style={opacity=0},apex={(\wX/2,-\h)}},postaction={draw}] (0,0) -- (\wX,0);}{\scoped[shift={(0,\h/2)}]\path[suspension={apex style={opacity=0},step=1.5cm/20,apex={(\wA/2,\h/2)}}] (\wA*.25,0) -- (\wA*.75,0);\fill[blue] (\wA/2,0) circle (1pt);}\&
\path[use as bounding box] (0,0) -- (\wX,0);\temporal<3-6>{}{\path[cone={step=1.5cm/10,apex style={opacity=0},apex={(\wA/2,\h)}},postaction={draw,A}] (0,0) -- (\wA,0);\path[cone={step=1.5cm/10,apex style={opacity=0},apex={(\wX/2,-\h)}},postaction={draw}] (0,0) -- (\wX,0);\path[cone={style={blue},step=1.5cm/10,apex style={opacity=0},apex={(\wX/2+\wA/2,\h*1.5)}}] (\wX,0) -- (0,0) -- (\wA/2,\h) -- (\wA,0);}{\scoped[shift={(0,-\h/2)}]\path[suspension={apex style={opacity=0},step=1.5cm/20,apex={(\wX/2,\h/2)}}] (\wX*.25,0) -- (\wX*.75,0);\fill[blue] (\wX/2,0) circle (1pt);}
\\
};
\end{tikzpicture}
\end{center}
\end{overlayarea}
\end{onlyenv}

\begin{onlyenv}<10->
Applicando il funtore $h^n$, otteniamo una successione di gruppi abeliani:
\begin{diagram}
\tikzmarknode{mark3}{\hmaps{A,K_n}}\&\hmaps{X,K_n}\lar\&\hmaps{X/A,K_n}\lar\&\hmaps{\Sigma A,K_n}\lar\&\hmaps{\Sigma X,K_n}\lar.
\end{diagram}
\tikz[overlay,remember picture] \node[blue,anchor=base,draw,circle,font=\footnotesize,inner sep=1pt,left=.3cm of  mark3] {\textbf{3}};

\begin{onlyenv}<11-16>
Per verificarne l'esattezza, è sufficiente mostrare che
\begin{diagram}
\hmaps{A,K_n}\&\hmaps{X,K_n}\lar\&\hmaps{X\cup\cone{A},K_n}\lar
\end{diagram}
è esatta.\end{onlyenv} \begin{onlyenv}<12-16>Sia $\map{f}{X}{K_n}$. Allora
\begingroup
\addtolength\jot{.15em}
\begin{align*}
&\onslide<13->{\text{$f$ appartiene al nucleo di $\umap{\hmaps{X,K_n}}{\hmaps{A,K_n}}$}}\\
\onslide<14->{\iff}&\onslide<14->{\text{$f|_A$ è omotopicamente banale (fissando il punto base)}}\\
\onslide<15->{\iff}&\onslide<15->{\text{$f$ si estende a una mappa $\umap{X\cup\cone{A}}{K_n}$}}\\
\onslide<16->{\iff}&\onslide<16->{\text{$f$ appartiene all'immagine di $\umap{\hmaps{X\cup\cone{A},K_n}}{\hmaps{X,K_n}}$}.}
\end{align*}\endgroup
\end{onlyenv}

\begin{onlyenv}<17->
Tale successione è esatta e naturale. Posto $K=K_n,K'=K_{n+1}$, le due successioni corrispondenti si possono incollare.
\begin{visibleenv}<18->
\begin{diagram}[column sep=10pt]
\&\&\hmaps{A,K}\dar{\iso}\ar[onslide=<19->,blue,ddl,bend right]\&\hmaps{X,K}\lar\dar{\iso}\&\hmaps{X/A,K}\lar\&\tikzmarknode{mark3 again}{\hmaps{\Sigma A,K}}\lar\\
\&\&\hmaps{A,\Omega K'}\dar{\iso}\&\hmaps{X,\Omega K'}\lar\dar{\iso}\\
\tikzmarknode{mark3 prime}{\hmaps{X,K'}}\&\hmaps{X/A,K'}\lar\&\hmaps{\Sigma A,K'}\lar\&\hmaps{\Sigma X,K'}\lar
\end{diagram}
\begin{tikzpicture}[overlay,remember picture,every node/.style={blue,anchor=base,draw,circle,font=\footnotesize,inner sep=1pt}]
\node[right=.3cm of  mark3 again] {\textbf{3}};
\node[left=.3cm of  mark3 prime] {\textbf{3'}};
\end{tikzpicture}
\end{visibleenv}

\begin{visibleenv}<20->
Otteniamo così la successione esatta lunga
\begin{diagram}[column sep=small]
\ldots\&\hmaps{X/A,K_{n+1}}\lar\&\hmaps{A,K_n}\lar\&\hmaps{X,K_n}\lar\&\hmaps{X/A,K_n}\lar\&\ldots\lar
\end{diagram}
naturale in $(X,A)$.
\end{visibleenv}
\end{onlyenv}
\end{onlyenv}
\end{overlayarea}
\end{frame*}
\section*{Teorie coomologiche}

\subsection*{Enunciato}
\begin{frame*}
\begin{theorem}
Siano $h^*$, $\bar{h}^*$ teorie coomologiche ridotte sulla categoria \CW0 tali che
\[
\begin{dcases*}
h^n(S^0)=\bar{h}^n(S^0)=0&per $n\neq 0$,\\
h^0(S^0)\iso \bar{h}^0(S^0)=G.
\end{dcases*}
\]
Allora per ogni $n\in\ZZ$ i funtori $h^n$ e $\bar{h}^n$ sono isomorfi.
\end{theorem}
\begin{remark}
Se $h$ è una teoria coomologica ridotta, è facile verificare che esiste un isomorfismo di funtori
\[
\map{h^n\iso h^{n+1}\circ\Sigma}{\CW0}{\Ab^{\text{op}}.}
\]
Di conseguenza possiamo limitarci a considerare CW complessi senza $1$-celle (infatti $\Sigma^2X$ ha una struttura di CW complesso senza $1$-celle).
\end{remark}
\end{frame*}

\subsection*{Dimostrazione}
\begin{frame*}
\begin{itemize}
\item La coomologia di $S^n$ è banale in ogni dimensione, tranne per
\[
h^n(S^n)=G.
\]
\item Sia $X$ un CW-complesso. Componendo le mappe
\begin{diagram}
h^n(X^n/X^{n-1})\rar\& h^n(X^n)\rar\&h^{n+1}(X^{n+1}/X^n)
\end{diagram}
si ottengono le applicazioni di bordo di un complesso ``cellulare''
\begin{diagram}[column sep=small]
\ldots\rar\&h^{n-1}(X^{n-1}/X^{n-2})\rar\&h^n(X^n/X^{n-1})\rar\&h^{n+1}(X^{n+1}/X^n)\rar\&\ldots,
\end{diagram}
il cui $n$-esimo gruppo di coomologia è precisamente $h^n(X)$.
\item Le mappe caratteristiche
\[
\map{\varphi_\alpha}{(D^n_\alpha,\partial D^n_\alpha)}{(X^n,X^{n-1})}
\]
inducono isomorfismi
\[
\umap[\iso]{h^n(X^n/X^{n-1})}{\textstyle\prod_\alpha h^n(D^n_\alpha/\partial D^n_\alpha)\iso\prod_\alpha G_\alpha.}
\]
\end{itemize}
\end{frame*}

\begin{frame*}
Dunque i gruppi del complesso cellulare non dipendono da $h^*$. Calcoliamo ora le applicazioni di bordo.

La mappa $\umap{G_\alpha}{G_\beta}$ data dalla composizione
\begin{diagram}[column sep=small]
G_\alpha\rar\dar[dash]{\iso}\&{\textstyle\prod_\alpha G_\alpha}\rar\dar[dash]{\iso}\&{\textstyle\prod_\beta G_\beta}\rar\dar[dash]{\iso}\&G_\beta\dar[dash]{\iso}\\
h^n(D^n_\alpha/\partial D^n_\alpha)\rar\&h^n(X^n/X^{n-1})\rar\&h^{n+1}(X^{n+1}/X^n)\rar\&h^{n+1}(D^{n+1}_\beta/\partial D^{n+1}_\beta)\dar[dash]{\iso}\\
\&\&\&h^n(\partial D^{n+1}_\beta)
\end{diagram}
è indotta dall'applicazione continua
\begin{diagram}
S^n\iso\partial D^{n+1}_\beta\rar[hook]\&X^n\rar\&D^n_\alpha/\partial D^n_\alpha\iso S^n.
\end{diagram}
\end{frame*}

\begin{frame}
content...
\end{frame}
\end{document}