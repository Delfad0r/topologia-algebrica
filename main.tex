\documentclass[9pt]{beamer}
\usepackage{mystyle}

\begin{document}

\newcommand*{\statemainresult}{%
\begin{theorem}
Siano $X$ un CW-complesso, $G$ un gruppo abeliano, $n\ge 0$ un intero. Allora esiste una biiezione
\[
\map[\iso]{T}{\tikzmarknode{mark}{\hmaps{X,K(G,n)}}}{\til{H}^n(X;G)}
\]
naturale in $X$.
\end{theorem}
}
\section*{Costruzione omotopica della coomologia}
\subsection*{Enunciato}
\begin{frame*}
\statemainresult%
\only<2>{\nointerlineskip\tikz[overlay,remember picture]\draw[overlay color] ($(mark.north east)+(1pt,3pt)$) rectangle ($(mark.south west)+(-1pt,-2pt)$);}
\pause\pause
\begin{itemize}[<+->]
\item Esiste una struttura canonica di gruppo abeliano su $\hmaps{X,K(G,n)}$ che rende $T$ un isomorfismo di gruppi.
\item $T$ è della forma
\[
T(f)=f^*(\alpha)
\]
dove $\alpha$ è la ``classe fondamentale'' di $\til{H}^n(K(G,n);G)$.
\end{itemize}
\end{frame*}

\subsection*{Strategia dimostrativa}
\begin{frame*}
\begin{enumerate}[<+->]
\addtolength\itemsep{1em}
\item Struttura di gruppo su $\hmaps{X,K(G,n)}$ $\Longrightarrow$ definizione di $\Omega$-spettro.
\item Per ogni $\Omega$-spettro $\{K_n\}$, la famiglia di funtori $h^n=\hmaps{-,K_n}$ è una teoria coomologica ridotta sulla categoria dei CW-complessi puntati.
\item Se una teoria coomologica ridotta $h^*$ soddisfa $h^n(S^0)=0$ per $n\neq 0$, allora esistono isomorfismi naturali $h^n(X)\iso\til{H}^n(X;h^0(S^0))$.
\end{enumerate}
\end{frame*}

%\section*{Aggiunzione \texorpdfstring{$\Sigma\dashv\Omega$}{Sigma-Omega}}
\subsection*{Categorie \texorpdfstring{\Top0}{Top.} e \texorpdfstring{\CW0}{CW.}}
\begin{frame*}
Lavoreremo nella categoria \Top0 e nella sua sottocategoria \CW0.
\begin{itemize}
\item Gli oggetti sono gli spazi topologici (o CW-complessi) puntati $(X,x_0)$.
\item I morfismi sono
\[
\Hom((X,x_0),(Y,y_0))=\hmaps{X,Y},
\]
funzioni continue $\map{f}{(X,x_0)}{(Y,y_0)}$ a meno di omotopia che fissa il punto base.
\item La composizione
\[
\map{{}\circ{}}{\hmaps{Y,Z}\times\hmaps{X,Y}}{\hmaps{X,Z}}
\]
è ben definita.
\end{itemize}
\begin{example}
Abbiamo l'uguaglianza (per ora solo insiemistica)
\[
\pi_n(X)=\hmaps{S^n,X}.
\]
\end{example}
\end{frame*}

\begin{comment}

\subsection*{Equivalenze omotopiche deboli}
\begin{frame*}
\begin{definition}
Una funzione $\map{f}{X}{Y}$ si dice \emph{equivalenza omotopica debole} se per ogni $n\ge 0$  la mappa indotta
\[
\map{f_*}{\pi_n(X)}{\pi_n(Y)}
\]
è un isomorfismo.
\end{definition}
\begin{proposition}
Siano $X$ un CW-complesso, $\map{f}{Y}{Z}$ un'equivalenza omotopica debole di spazi topologici. Allora
\[
\map{f\circ-}{\hmaps{X,Y}}{\hmaps{X,Z}}
\]
è una biiezione.
\end{proposition}
\begin{theorem}[Approssimazione CW]
Per ogni spazio topologico $X$ esistono un CW-complesso $Z$ e un'equivalenza omotopica debole
$\map{f}{Z}{X}$.
\end{theorem}
\end{frame*}
\end{comment}

\subsection*{Sospensione}
\begin{frame*}
\begin{itemize}
\item La sospensione $SX$ di uno spazio topologico $X$ è
\begin{align*}
SX=X\times[0,1]/\sim,&&\text{$X\times\{0\}$ e $X\times\{1\}$ collassati a due punti.}
\end{align*}
\item Per uno spazio puntato $(X,x_0)$, è conveniente considerare la sospensione ridotta
\begin{align*}
\Sigma X=SX/\{x_0\}\times[0,1],&&\text{con punto base $[x_0]$.}
\end{align*}
\item Ogni $f\in\hmaps{X,Y}$ induce $\Sigma f\in\hmaps{\Sigma X,\Sigma Y}$:
\[
\Sigma f(x,t)=(f(x),t).
\]
\item La sospensione ridotta è dunque un funtore
\[
\map{\Sigma}{\Top0}{\Top0}
\]
che si restringe a
\[
\map{\Sigma}{\CW0}{\CW0}.
\]
\end{itemize}
\end{frame*}

\begin{frame*}
Dati due spazi topologici puntati $X$, $Y$, esiste una struttura di gruppo canonica su $\hmaps{\Sigma X,Y}$.
\begin{itemize}
\item Per $f,g\in\hmaps{\Sigma X,Y}$, il prodotto $f\cdot g$ è così definito:
\[
f\cdot g(x,t)=\begin{dcases}f(x,2t)&t\le\frac{1}{2},\\g(x,2t-1)&t\ge\frac{1}{2}.\end{dcases}
\]
\item Si verifica facilmente che questa operazione definisce una struttura di gruppo, avente la funzione costante come elemento neutro.
\item Per ogni $\map{f}{Y}{Z}$, la composizione
\[
\map{f\circ-{}}{\hmaps{\Sigma X,Y}}{\hmaps{\Sigma X,Z}}
\]
 è un omomorfismo di gruppi.
 \item Di conseguenza, otteniamo il funtore
 \[
 \map{\hmaps{\Sigma X,-}}{\Top0}{\Grp}.
 \]
\end{itemize}
\end{frame*}

\subsection*{Spazio di lacci}
\begin{frame*}
\begin{itemize}
\item Dato uno spazio topologico puntato $(X,x_0)$, definiamo lo spazio di lacci
\[
\Omega X=\left\{g\in X^{[0,1]}:g(0)=g(1)=x_0\right\}
\]
con punto base il cammino costante in $x_0$. La topologia è quella compatta-aperta.
\item Ogni $f\in\hmaps{X,Y}$ induce $\Omega f\in\hmaps{\Omega X,\Omega Y}$ per composizione:
\[
\Omega f(g)=f\circ g.
\]
 \item Lo spazio di lacci è dunque un funtore
\[
\map{\Omega}{\Top0}{\Top0}.
\]
\end{itemize}
\end{frame*}

\begin{frame*}
In modo duale a quanto accade per la sospensione, dati due spazi topologici puntati $X$, $Y$, esiste una struttura di gruppo canonica su $\hmaps{X,\Omega Y}$.
\begin{itemize}
\item Per $f,g\in\hmaps{X,\Omega Y}$, il prodotto $f\cdot g$ è così definito:
\[
f\cdot g(x)(t)=\begin{dcases}f(x)(2t)&t\le\frac{1}{2},\\g(x)(2t-1)&t\ge\frac{1}{2}.\end{dcases}
\]
\item Si verifica facilmente che questa operazione definisce una struttura di gruppo, avente la funzione costante come elemento neutro.
\item Per ogni $\map{f}{X}{Z}$, la composizione
\[
\map{-\circ f}{\hmaps{Z,\Omega Y}}{\hmaps{X,\Omega Y}}
\]
 è un omomorfismo di gruppi.
 \item Di conseguenza, otteniamo il funtore
 \[
 \map{\hmaps{-,\Omega Y}}{\Top0}{\Grp^{\text{op}}}.
 \]
\end{itemize}
\end{frame*}

\subsection*{Osservazioni}
\begin{frame*}
\begin{itemize}
\item Con la struttura di gruppo che abbiamo definito, l'uguaglianza
\[
\pi_n(X)=\hmaps{S^n,X}
\]
è un isomorfismo di gruppi.
\item Le due strutture di gruppo su  $\hmaps{\Sigma X,\Omega Y}$ coincidono.
\item $\Omega^2Y$ può essere interpretato come lo spazio delle funzioni
\[
\umap{[0,1]\times[0,1]}{Y}
\]
che assumono il valore $y_0$ sul bordo. Con lo stesso ragionamento usato per il $\pi_2$, si dimostra che $\hmaps{X,\Omega^2Y}$ è abeliano.
\end{itemize}
\end{frame*}

\subsection*{Aggiunzione}
\begin{frame*}
\begin{proposition}
I funtori $\Sigma$ e $\Omega$ sono aggiunti. In altre parole, esistono biiezioni
\[
\umap[\iso]{\hmaps{\Sigma X,Y}}{\hmaps{X,\Omega Y}}
\]
naturali in $X$ e $Y$. Inoltre queste biiezioni sono isomorfismi di gruppi.
\end{proposition}
\begin{itemize}
\item Prendendo $X=S^{n-1}$, otteniamo che
\[
\pi_n(X)=\hmaps{S^n,X}=\hmaps{\Sigma S^{n-1},X}\iso\hmaps{S^{n-1},\Omega X}=\pi_{n-1}(\Omega X).
\]
\item In particolare, se $X$ è un $K(G,n)$, allora $\Omega X$ è un $K(G,n-1)$.
\item Per ogni $n\ge 1$, sia $K_n$ un CW-complesso che è anche un $K(G,n)$; per unicità dei $K(G,n)$, abbiamo equivalenze omotopiche deboli
\[
\map{\theta_n}{K_n}{\Omega K_{n+1}}.
\]
\end{itemize}
\end{frame*}
\section*{\texorpdfstring{$\Omega$}{Omega}-spettri e coomologia}

\subsection*{Definizione}
\begin{frame*}
\begin{definition}
Un \emph{$\Omega$-spettro} è una famiglia di CW-complessi $\{K_n\}_{n\ge 1}$ dotata di equivalenze omotopiche deboli
\[
\map{\theta_n}{K_n}{\Omega K_{n+1}}.
\]
\end{definition}
\begin{itemize}
\item È possibile estendere la famiglia anche a indici $n\le 0$: è sufficiente considerare come $K_{n-1}$ un'approssimazione CW di $\Omega K_n$.
\end{itemize}
\begin{proposition}
Siano $\map{f}{Y}{Z}$ un'equivalenza omotopica debole, $X$ un CW-complesso. Allora la composizione
\[
\map[\iso]{f\circ-{}}{\hmaps{X,Y}}{\hmaps{X,Z}}
\]
è una biiezione.
\end{proposition}
\end{frame*}

\subsection*{Struttura di gruppo}
\begin{frame*}
D'ora in poi, tutti gli spazi di cui parleremo saranno CW-complessi puntati.
\begin{itemize}
\item Possiamo dare a $\hmaps{X,K_n}$ una struttura di gruppo imponendo che
\[
\map[\iso]{\theta_n\circ-{}}{\hmaps{X,K_n}}{\hmaps{X,\Omega K_{n+1}}}
\]
sia un isomorfismo di gruppi.
\item In questo modo, $\hmaps{X,K_n}$ risulta essere un gruppo abeliano:
\begin{diagram}[column sep=huge]
\&\hmaps{\Sigma X,K_{n+1}}\rar["\iso"']{\theta_{n+1}\circ-}\dar{\iso}\&\hmaps{\Sigma X,\Omega K_{n+2}}\dar{\iso}\\
\hmaps{X,K_n}\rar["\iso"']{\theta_n\circ-}\&\hmaps{X,\Omega K_{n+1}}\rar{\Omega\theta_{n+1}\circ-}\&\hmaps{X,\Omega^2 K_{n+2}}\nospacepunct{.}
\end{diagram}
\item Di conseguenza, per ogni $n\in\ZZ$ otteniamo il funtore
\[
\map{\hmaps{-,K_n}}{\CW0}{\Ab^{\text{op}}}.
\]
\end{itemize}
\end{frame*}

\subsection*{Teorie coomologiche}
\begin{frame*}
\begin{theorem}
Sia $\{K_n\}_{n\in\ZZ}$ un $\Omega$-spettro. Allora i funtori $h^n=\hmaps{-,K_n}$ definiscono una teoria coomologica ridotta sulla categoria \CW0.
\end{theorem}
\begin{reminder}
Una \emph{teoria coomologica ridotta} sulla categoria \CW0 è una famiglia di funtori
\begin{align*}
\map{h^n}{\CW0}{\Ab^{\text{op}}},&&n\in\ZZ
\end{align*}
che soddisfa i seguenti assiomi.
\begin{enumerate}
\item\label{cohomology-axiom-les} Per ogni coppia $(X,A)$ con $x_0\in A$ esiste una successione esatta lunga
\begin{diagram}[column sep=small]
\ldots\rar{\delta}\&h^n(X/A)\rar{q^*}\&h^n(X)\rar{i^*}\&h^n(A)\rar{\delta}\&h^{n+1}(X/A)\rar{q^*}\&\ldots
\end{diagram}
naturale nella coppia $(X,A)$.
\item\label{cohomology-axiom-wedge} Per ogni famiglia $\{X_\alpha\}_{\alpha}$, le inclusioni inducono un isomorfismo
\[
\umap[\iso]{h^n\left(\textstyle\bigvee_\alpha X_\alpha\right)}{\textstyle\prod_\alpha h^n(X_\alpha)}.
\]
\end{enumerate}
\end{reminder}
\end{frame*}

\subsection*{Dimostrazione}
\begin{frame*}
\begin{itemize}
\item L'assioma \ref{cohomology-axiom-wedge}. è soddisfatto: nella categoria \CW0 dare un morfismo
\[
\umap{{\textstyle\bigvee_\alpha X_\alpha}}{K_n}
\]
è equivalente a dare  una collezione di morfismi
\[
\{\umap{X_\alpha}{K_n}\}_\alpha.
\]
\item Per quanto riguarda l'assioma \ref{cohomology-axiom-les}., sia $(X,A)$ una coppia di CW-complessi; vediamo come costruire la successione esatta lunga associata.
\end{itemize}
\end{frame*}

\begin{frame*}
\begin{onlyenv}<1>
\begin{diagram}[column sep=11pt]
A\rar[hook]\dar[equal]\&X\rar[hook]\dar[equal]\&X\cup\cone{A}\rar[hook]\dar{\iso}\&(X\cup\cone{A})\cup\cone{X}\rar[hook]\dar{\iso}\&\big((X\cup\cone{A})\cup\cone{X}\big)\cup\cone(X\cup\cone{A})\dar{\iso}\\
A\rar[hook]\dar\&X\rar\dar\&X/A\rar\dar\&\Sigma A\rar[hook]\dar\&\Sigma X\dar\\
B\rar[hook]\&Y\rar\&Y/B\rar\&\Sigma B\rar[hook]\&\Sigma Y
\end{diagram}
\end{onlyenv}

Applicando il funtore $h^n$, otteniamo una successione di gruppi abeliani:
\begin{diagram}
\hmaps{A,K_n}\&\hmaps{X,K_n}\lar\&\hmaps{X/A,K_n}\lar\&\hmaps{\Sigma A,K_n}\lar\&\hmaps{\Sigma X,K_n}\lar.
\end{diagram}

\begin{onlyenv}<1-2>
Per verificarne l'esattezza, è sufficiente mostrare che
\begin{diagram}
\hmaps{A,K_n}\&\hmaps{X,K_n}\lar\&\hmaps{X\cup\cone{A},K_n}\lar
\end{diagram}
è esatta.\end{onlyenv} \begin{onlyenv}<2>Sia $\map{f}{X}{K_n}$. Allora
\begingroup
\addtolength\jot{.15em}
\begin{align*}
&\text{$f$ appartiene al nucleo di $\umap{\hmaps{X,K_n}}{\hmaps{A,K_n}}$}\\
\iff&\text{$f|_A$ è omotopicamente banale (fissando il punto base)}\\
\iff&\text{$f$ si estende a una mappa $\umap{X\cup\cone{A}}{K_n}$}\\
\iff&\text{$f$ appartiene all'immagine di $\umap{\hmaps{X\cup\cone{A},K_n}}{\hmaps{X,K_n}}$}.
\end{align*}
\endgroup
\end{onlyenv}

\begin{onlyenv}<3>
Tale successione è esatta e naturale. Posto $K=K_n,K'=K_{n+1}$, le due successioni corrispondenti si possono incollare.
\begin{diagram}[column sep=10pt]
\&\&\hmaps{A,K}\dar{\iso}\&\hmaps{X,K}\lar\dar{\iso}\&\hmaps{X/A,K}\lar\&\hmaps{\Sigma A,K}\lar\\
\&\&\hmaps{A,\Omega K'}\dar{\iso}\&\hmaps{X,\Omega K'}\lar\dar{\iso}\\
\hmaps{X,K'}\&\hmaps{X/A,K'}\lar\&\hmaps{\Sigma A,K'}\lar\&\hmaps{\Sigma X,K'}\lar
\end{diagram}
Otteniamo così la successione esatta lunga
\begin{diagram}[column sep=small]
\ldots\&\hmaps{X/A,K_{n+1}}\lar\&\hmaps{A,K_n}\lar\&\hmaps{X,K_n}\lar\&\hmaps{X/A,K_n}\lar\&\ldots\lar
\end{diagram}
naturale in $(X,A)$.
\end{onlyenv}
\end{frame*}
%\section*{Teorie coomologiche}

\subsection*{Enunciato}
\begin{frame*}
\begin{theorem}
Siano $h^*,\bar{h}^*$ teorie coomologiche ridotte sulla categoria \CW0 tali che
\[
\begin{dcases*}
h^n(S^0)=\bar{h}^n(S^0)=0&per $n\neq 0$,\\
h^0(S^0)\iso \bar{h}^0(S^0)=G.
\end{dcases*}
\]
Allora per ogni $n\in\ZZ$ i funtori $h^n$ e $\bar{h}^n$ sono isomorfi.
\end{theorem}
\begin{remark}
Se $h$ è una teoria coomologica ridotta, è facile verificare che esiste un isomorfismo di funtori
\[
\map{h^n\iso h^{n+1}\circ\Sigma}{\CW0}{\Ab^{\text{op}}.}
\]
Di conseguenza possiamo limitarci a considerare CW complessi senza $1$-celle (infatti $\Sigma^2X$ ha una struttura di CW complesso senza $1$-celle).
\end{remark}
\end{frame*}

\subsection*{Dimostrazione}
\begin{frame*}
\begin{itemize}
\item La coomologia di $S^n$ è banale in ogni dimensione, tranne per
\[
h^n(S^n)=G.
\]
\item Sia $X$ un CW-complesso. Componendo le mappe
\begin{diagram}
h^n(X^n/X^{n-1})\rar\& h^n(X^n)\rar\&h^{n+1}(X^{n+1}/X^n)
\end{diagram}
si ottengono le applicazioni di bordo di un complesso ``cellulare''
\begin{diagram}[column sep=10pt]
\ldots\rar\&h^{n-1}(X^{n-1}/X^{n-2})\rar\&h^n(X^n/X^{n-1})\rar\&h^{n+1}(X^{n+1}/X^n)\rar\&\ldots,
\end{diagram}
il cui $n$-esimo gruppo di coomologia è precisamente $h^n(X)$.
\item Le mappe caratteristiche
\[
\map{\varphi_\alpha}{(D^n_\alpha,\partial D^n_\alpha)}{(X^n,X^{n-1})}
\]
inducono isomorfismi
\[
\umap[\iso]{h^n(X^n/X^{n-1})}{\textstyle\prod_\alpha h^n(D^n_\alpha/\partial D^n_\alpha)\iso\prod_\alpha G_\alpha.}
\]
\end{itemize}
\end{frame*}

\begin{frame*}
Dunque il complesso cellulare è della forma
\begin{diagram}[column sep=10pt]
\ldots\rar\&{\textstyle\prod_{\bar\alpha} G_{\bar\alpha}}\rar\ar[dd,dash,"\iso"]\&{\textstyle\prod_{\bar\beta} G_{\bar\beta}}\ar[dd,dash,"\iso"]\rar\ar[rd]\&\ldots\\
G_\alpha\ar[ru]\dar[dash]{\iso}\&\&\&G_\beta\dar[dash]{\iso}\\
h^n(D^n_\alpha/\partial D^n_\alpha)\rar\&h^n(X^n/X^{n-1})\rar\&h^{n+1}(X^{n+1}/X^n)\rar\&h^{n+1}(D^{n+1}_\beta/\partial D^{n+1}_\beta)\dar[dash]{\iso}\\
\&\&\&h^n(\partial D^{n+1}_\beta)\nospacepunct{.}
\end{diagram}
La mappa $\umap{G_\alpha}{G_\beta}$ è indotta dall'applicazione continua
\begin{diagram}
S^n\iso\partial D^{n+1}_\beta\rar\&X^n\rar\&D^n_\alpha/\partial D^n_\alpha\iso S^n.
\end{diagram}
\end{frame*}

\begin{frame*}
Mostriamo ora che l'applicazione 
\[
\map{f^*}{h^n(S^n)}{h^n(S^n)}
\]
indotta in coomologia da una funzione continua
\[
\map{f}{S^n}{S^n}
\]
è la moltiplicazione per il grado di $f$.
\begin{itemize}
\item Sappiamo che due mappe $\umap{S^n}{S^n}$ sono omotope se e solo se hanno lo stesso grado.
\item L'enunciato è vero per $\deg(f)=0,1$ (funzione costante e identità).
\item È sufficiente mostrare che $(f\cdot g)^*=f^*+g^*$.
\end{itemize}
\begin{lemma}
Siano $X,Y$ CW complessi, $f,g\in\hmaps{\Sigma X,Y}$. Allora
\[
(f\cdot g)^*=f^*+g^*.
\]
\end{lemma}
\end{frame*}

\begin{frame*}
Un omomorfismo
\[
\umap{{\textstyle\prod_{\bar\alpha}G_{\bar\alpha}}}{{\textstyle\prod_{\bar\beta}G_{\bar\beta}}}
\]
non è determinato dai valori che assume sui singoli fattori $G_{\bar\alpha}$. Tuttavia per ogni $\beta$ la composizione
\begin{diagram}
{\textstyle\prod_{\bar\alpha}G_{\bar\alpha}}\rar\dar\&{\textstyle\prod_{\bar\beta}G_{\bar\beta}}\rar\&G_\beta\iso h^n(\partial D^{n+1}_\beta)\\
{\textstyle\prod_{\bar\gamma}G_{\bar\gamma}}\ar[urr]
\end{diagram}
è indotta da
\begin{diagram}
X^n/X^{n-1}\&X^n\lar\&\partial D^{n+1}_\beta\lar\ar[dl]\\
X^n_\beta/X^{n-1}\uar[hook]\&X^n_\beta\lar\uar[hook],
\end{diagram}
dove $X^n_\beta$ si ottiene attaccando a $X^{n-1}$ le $n$-celle che intersecano l'immagine di $\partial D^{n+1}_\beta$ (sono in numero finito).
\end{frame*}

\section*{Costruzione omotopica della coomologia}
\subsection*{Dimostrazione}
\begin{frame*}
\statemainresult
\pause
\begin{itemize}[<+->]
\item Poniamo $K_n=K(G,n)$; sappiamo che è un $\Omega$-spettro.
\item Estendendo la successione a $n\le 0$, otteniamo la famiglia di funtori
\[
\map{h^n=\hmaps{-,K_n}}{\CW0}{\Ab^{\text{op}}}
\]
che definisce una teoria coomologica ridotta.
\item Vale
\[
h^n(S^0)=\til{H}^n(S^0;G)=\begin{dcases}0&n\neq0,\\G&n=0.\end{dcases}
\]
\item Di conseguenza, per ogni $n$ abbiamo un isomorfismo di funtori
\[
h^n\iso\til{H}^n(-;G).
\]
\end{itemize}
\end{frame*}

\subsection*{Classe fondamentale}
\begin{frame*}
Per il lemma di Yoneda, l'isomorfismo
\[
\map[\iso]{T}{\hmaps{X,K(G,n)}}{\til{H}^n(X;G)}
\]
è della forma
\[
T(f)=f^*(\alpha)\in\til{H}^n(X;G)
\]
dove
\[
\alpha=T(\id_{K(G,n)})\in\til{H}^n(K(G,n);G).
\]
\pause
\begin{itemize}[<+->]
\item Se prendiamo $K(G,n)$ in modo che abbia come $(n-1)$-scheletro un punto, $\alpha$ è rappresentata dal cociclo cellulare che a ogni $n$-cella associa l'elemento
\[
[\map{\varphi_\alpha}{D^n_\alpha/\partial D^n_\alpha}{K(G,n)}]\in\pi_n(K(G,n))\iso G
\]
indotto dalla mappa caratteristica.
\item Analogamente, per ogni $\map{f}{X}{K(G,n)}$, $T(f)$ è rappresentato in coomologia dal cociclo cellulare che a ogni $n$-cella di $X$ associa l'elemento di $\pi_n(K(G,n))$ indotto dalla composizione
\begin{diagram}
D^n_\alpha/\partial D^n_\alpha\rar{\varphi_\alpha}\&X^n\rar{f}\&K(G,n).
\end{diagram}
\end{itemize}
\end{frame*}

\end{document}