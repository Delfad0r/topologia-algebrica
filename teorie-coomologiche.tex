\section*{Teorie coomologiche}

\subsection*{Enunciato}
\begin{frame*}
\begin{theorem}
Siano $h^*,\bar{h}^*$ teorie coomologiche ridotte sulla categoria \CW0 tali che
\[
\begin{dcases*}
h^n(S^0)=\bar{h}^n(S^0)=0&per $n\neq 0$,\\
h^0(S^0)\iso \bar{h}^0(S^0)=G.
\end{dcases*}
\]
Allora per ogni $n\in\ZZ$ i funtori $h^n$ e $\bar{h}^n$ sono isomorfi.
\end{theorem}
\pause
\begin{remark}
Se $h$ è una teoria coomologica ridotta, è facile verificare che esiste un isomorfismo di funtori
\[
\map{h^n\iso h^{n+1}\circ\Sigma}{\CW0}{\Ab^{\text{op}}.}
\]
\pause
Di conseguenza possiamo limitarci a considerare CW complessi senza $1$-celle (infatti $\Sigma^2X$ ha una struttura di CW complesso senza $1$-celle).
\end{remark}
\end{frame*}

\subsection*{Dimostrazione}
\begin{frame*}
\begin{itemize}[<+->]
\item La coomologia di $S^n$ è banale in ogni dimensione, tranne per
\[
h^n(S^n)=G.
\]
\item Sia $X$ un CW-complesso. Componendo le mappe
\begin{diagram}
h^n(X^n/X^{n-1})\rar\& h^n(X^n)\rar\&h^{n+1}(X^{n+1}/X^n)
\end{diagram}
si ottengono le applicazioni di bordo di un complesso ``cellulare''
\begin{diagram}[column sep=10pt]
\ldots\rar\&h^{n-1}(X^{n-1}/X^{n-2})\rar\&h^n(X^n/X^{n-1})\rar\&h^{n+1}(X^{n+1}/X^n)\rar\&\ldots,
\end{diagram}
il cui $n$-esimo gruppo di coomologia è precisamente $h^n(X)$.
\item Le mappe caratteristiche
\[
\map{\varphi_\alpha}{(D^n_\alpha,\partial D^n_\alpha)}{(X^n,X^{n-1})}
\]
inducono isomorfismi
\[
\umap[\iso]{h^n(X^n/X^{n-1})}{\textstyle\prod_\alpha h^n(D^n_\alpha/\partial D^n_\alpha)\iso\prod_\alpha G_\alpha.}
\]
\end{itemize}
\end{frame*}

\begin{frame*}
Dunque il complesso cellulare è della forma
\begin{diagram}[column sep=10pt]
\ldots\rar\&{\textstyle\prod_\alpha G_\alpha}\rar{d}\ar[dd,dash,"\iso"]\&{\textstyle\prod_\beta G_\beta}\ar[dd,dash,"\iso"]\rar\ar[onslide=<3->,rd]\&\ldots\\
{\visible<2->{G_\alpha}}\ar[onslide=<2->,ru]\dar[onslide=<2->,dash]{\iso}\ar[onslide=<4->,blue,bend left=20,rrr]\&\&\&{\visible<3->{G_\beta}}\dar[onslide=<3->,dash]{\iso}\\
{\visible<2->{h^n(D^n_\alpha/\partial D^n_\alpha)}}\rar[onslide=<2->]\&h^n(X^n/X^{n-1})\rar{d}\&h^{n+1}(X^{n+1}/X^n)\rar[onslide=<3->]\&{\visible<3->{h^{n+1}(D^{n+1}_\beta/\partial D^{n+1}_\beta)}}\dar[onslide=<3->,dash]{\iso}\\
\&\&\&{\visible<3->{h^n(\partial D^{n+1}_\beta)\nospacepunct{.}}}
\end{diagram}
\begin{visibleenv}<4->
La mappa $\umap{G_\alpha}{G_\beta}$ è indotta dall'applicazione continua
\begin{diagram}
{\visible<5->{S^n\iso}}\partial D^{n+1}_\beta\rar\&X^n\rar\&D^n_\alpha/\partial D^n_\alpha{\visible<5->{\iso S^n.}}
\end{diagram}
\end{visibleenv}
\end{frame*}

\begin{frame*}
Mostriamo ora che l'applicazione 
\[
\map{f^*}{h^n(S^n)}{h^n(S^n)}
\]
indotta in coomologia da una funzione continua
\[
\map{f}{S^n}{S^n}
\]
è la moltiplicazione per il grado di $f$.
\pause
\begin{itemize}[<+->]
\item Sappiamo che due mappe $\umap{S^n}{S^n}$ sono omotope se e solo se hanno lo stesso grado.
\item L'enunciato è vero per $\deg(f)=0,1$ (funzione costante e identità).
\item È sufficiente mostrare che $(f+g)^*=f^*+g^*$.
\end{itemize}
\begin{lemma}<+->
Siano $X,Y$ CW complessi, $f,g\in\hmaps{\Sigma X,Y}$. Allora
\[
(f\cdot g)^*=f^*+g^*.
\]
\end{lemma}
\end{frame*}

\begin{frame*}
\visible<2->{Un omomorfismo}
\begin{diagram}
|[alt=<2-6>{gray}{}]|G_\alpha\rar[alt=<2-6>{gray}{}]\&{\textstyle\prod_\alpha G_\alpha}\rar{d}\&{\textstyle\prod_\beta G_\beta}\rar\&G_\beta
\end{diagram}

\begin{visibleenv}<2->
\alt<7>{{\color{gray}non}}{non} è determinato dalle restrizioni ai singoli fattori $G_\alpha$. \visible<3->{Tuttavia per ogni $\beta$ la composizione
\begin{diagram}
{\textstyle\prod_\alpha G_\alpha}\rar{d}\dar[onslide=<6->,blue]\&{\textstyle\prod_\beta G_\beta}\rar\&G_\beta\iso h^n(\partial D^{n+1}_\beta)\\
\tikzmarknode[onslide=<6->,blue]{mark1}{\prod_\gamma G_\gamma}\ar[onslide=<6->,blue,urr]
\end{diagram}}
\visible<4->{è indotta da
\begin{diagram}
X^n/X^{n-1}\&X^n\lar\&\partial D^{n+1}_\beta\lar\ar[onslide=<5->,blue,dl]\\
|[onslide=<5->,blue]|X^n_\beta/X^{n-1}\uar[onslide=<5->,blue,hook]\&|[onslide=<5->,blue]|X^n_\beta\lar[onslide=<5->,blue]\uar[onslide=<5->,blue,hook],
\end{diagram}}
\visible<5->{dove $X^n_\beta$ si ottiene attaccando a $X^{n-1}$ le $n$-celle che intersecano l'immagine di $\partial D^{n+1}_\beta$ (sono in numero finito).}
\end{visibleenv}
\end{frame*}

\begin{frame*}
Riassumendo:\pause
\begin{itemize}[<+->]
\item possiamo calcolare i gruppi $h^n(X)$ a partire dal complesso cellulare
%\begin{diagram}
%\ldots\rar\&{\textstyle\prod_\alpha G_\alpha}\rar\&{\textstyle\prod_\beta G_\beta}\rar\&\ldots;
%\end{diagram}
\begin{diagram}[column sep=10pt]
\ldots\rar\&h^{n-1}(X^{n-1}/X^{n-2})\rar\&h^n(X^n/X^{n-1})\rar\&h^{n+1}(X^{n+1}/X^n)\rar\&\ldots;
\end{diagram}
\item i gruppi di questo complesso dipendono solo dalla struttura di CW-complesso di $X$;
\item le mappe di bordo sono determinate dal grado di certe funzioni continue
\[
\umap{\partial D^{n+1}_\beta}{D^n_\alpha/\partial D^n_\alpha}
\]
che dipendono dalle mappe caratteristiche delle celle.
\end{itemize}

\visible<+>{Di conseguenza, teorie coomologiche diverse hanno lo stesso complesso cellulare, dunque per ogni $X$ abbiamo un isomorfismo
\[
h^n(X)\iso\bar{h}^n(X).
\]}
\end{frame*}

\begin{frame*}
Per verificare la naturalità, sia $\map{f}{X}{Y}$, cellulare a meno di omotopia.\pause
\begin{itemize}[<+->]
\item Vale $f(X^n)\subs Y^n$, dunque $f$ induce un morfismo fra i complessi cellulari
\begin{diagram}
\ldots\rar\&h^n(Y^n/Y^{n-1})\rar\dar{f^*}\&h^{n+1}(Y^{n+1}/Y^n)\rar\dar{f^*}\&\ldots\\
\ldots\rar\&h^n(X^n/X^{n-1})\rar\&h^{n+1}(X^{n+1}/X^n)\rar\&\ldots
\end{diagram}
\item Come accadeva per le mappe di bordo, i morfismi $f^*$ sono determinati dal grado delle applicazioni
\begin{diagram}
D^n_\alpha/\partial D^n_\alpha\rar\&X^n/X^{n-1}\rar{f}\&Y^n/Y^{n-1}\rar\&D^n_\beta/\partial D^n_\beta.
\end{diagram}
\end{itemize}
\visible<+>{\vspace{-.4cm}
Di conseguenza, a meno dell'isomorfismo che abbiamo stabilito fra i complessi cellulari di $h^*$ e $\bar{h}^*$, $f$ induce lo stesso morfismo di complessi nelle due teorie coomologiche. Pertanto l'isomorfismo
\[
h^n(X)\iso\bar{h}^n(X)
\]
è naturale in $X$.}
\end{frame*}