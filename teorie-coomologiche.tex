\section*{Teorie coomologiche}

\subsection*{Enunciato}
\begin{frame*}
\begin{theorem}
Siano $h^*$, $\bar{h}^*$ teorie coomologiche ridotte sulla categoria \CW0 tali che
\[
\begin{dcases*}
h^n(S^0)=\bar{h}^n(S^0)=0&per $n\neq 0$,\\
h^0(S^0)\iso \bar{h}^0(S^0)=G.
\end{dcases*}
\]
Allora per ogni $n\in\ZZ$ i funtori $h^n$ e $\bar{h}^n$ sono isomorfi.
\end{theorem}
\begin{remark}
Se $h$ è una teoria coomologica ridotta, è facile verificare che esiste un isomorfismo di funtori
\[
\map{h^n\iso h^{n+1}\circ\Sigma}{\CW0}{\Ab^{\text{op}}.}
\]
Di conseguenza possiamo limitarci a considerare CW complessi senza $1$-celle (infatti $\Sigma^2X$ ha una struttura di CW complesso senza $1$-celle).
\end{remark}
\end{frame*}

\subsection*{Dimostrazione}
\begin{frame*}
\begin{itemize}
\item La coomologia di $S^n$ è banale in ogni dimensione, tranne per
\[
h^n(S^n)=G.
\]
\item Sia $X$ un CW-complesso. Componendo le mappe
\begin{diagram}
h^n(X^n/X^{n-1})\rar\& h^n(X^n)\rar\&h^{n+1}(X^{n+1}/X^n)
\end{diagram}
si ottengono le applicazioni di bordo di un complesso ``cellulare''
\begin{diagram}[column sep=small]
\ldots\rar\&h^{n-1}(X^{n-1}/X^{n-2})\rar\&h^n(X^n/X^{n-1})\rar\&h^{n+1}(X^{n+1}/X^n)\rar\&\ldots,
\end{diagram}
il cui $n$-esimo gruppo di coomologia è precisamente $h^n(X)$.
\item Le mappe caratteristiche
\[
\map{\varphi_\alpha}{(D^n_\alpha,\partial D^n_\alpha)}{(X^n,X^{n-1})}
\]
inducono isomorfismi
\[
\umap[\iso]{h^n(X^n/X^{n-1})}{\textstyle\prod_\alpha h^n(D^n_\alpha/\partial D^n_\alpha)\iso\prod_\alpha G_\alpha.}
\]
\end{itemize}
\end{frame*}

\begin{frame*}
Dunque i gruppi del complesso cellulare non dipendono da $h^*$. Calcoliamo ora le applicazioni di bordo.

La mappa $\umap{G_\alpha}{G_\beta}$ data dalla composizione
\begin{diagram}[column sep=small]
G_\alpha\rar\dar[dash]{\iso}\&{\textstyle\prod_\alpha G_\alpha}\rar\dar[dash]{\iso}\&{\textstyle\prod_\beta G_\beta}\rar\dar[dash]{\iso}\&G_\beta\dar[dash]{\iso}\\
h^n(D^n_\alpha/\partial D^n_\alpha)\rar\&h^n(X^n/X^{n-1})\rar\&h^{n+1}(X^{n+1}/X^n)\rar\&h^{n+1}(D^{n+1}_\beta/\partial D^{n+1}_\beta)\dar[dash]{\iso}\\
\&\&\&h^n(\partial D^{n+1}_\beta)
\end{diagram}
è indotta dall'applicazione continua
\begin{diagram}
S^n\iso\partial D^{n+1}_\beta\rar[hook]\&X^n\rar\&D^n_\alpha/\partial D^n_\alpha\iso S^n.
\end{diagram}
\end{frame*}

\begin{frame}
content...
\end{frame}